% Conventions:
% 	- Use italics when refering to variable or function names in our code
%	- For todo's we use the todo comand

\documentclass{article}
\usepackage[margin=1in]{geometry}
\usepackage{xcolor}
\usepackage{hyperref}
\usepackage{parskip}

\hypersetup{
	colorlinks=true,
	urlcolor=blue
}
\newcommand{\todo}[1]{\textbf{\textcolor{red}{Todo: #1}}}



\title{Report on \href{https://github.com/tvmaaren/GeoAlgBonus-Yanidsmas}{Yanidsmas}}
\author{Yanna \todo{surname} (\todo{student number}), Ids de Vlas (6967396), Thomas van Maaren (9825827)}


\begin{document}

\maketitle

In this report we will discuss the programme Yanidsmas. This is a programme that can
given any set of points in the plane, calculate the vertices of it's convex hull. The
user has to choose between four algorithms: Graham Scan, Jarvis March, Divide
and Conquer and Chan's Algorithm. The programme can be cloned be from our
\href{https://github.com/tvmaaren/GeoAlgBonus-Yanidsmas}{git repository}.

\section*{Language and Environment}

We have decided to use Rust to write Yanidsmas. This is because we wanted an imperative
language that gives a very low chance of runtime errors or unpredictable behaviour.
We recommend to use cargo to run our programme.
Our git repository contains a \href{https://github.com/tvmaaren/GeoAlgBonus-Yanidsmas/tree/main/tests}{tests} directory that contains all the test cases we came up with.
When you run the program on a testcase it will output a png image. This image shows
a scatter plot of the points in the test case where every vertex of the Convex hull is
marked red. We did this to make it possible to see at a glance if the programme gives
the right output. When benchmarking the scatter plot is disabled as the
time this takes is not relevant to measure the speed of our algorithms. For a more
detailed explanation on how to use Yanidsmas we refer to README.md file in our git repository.

\section*{Correctness of the algorithms}

For every implemented algorithm we will give a short explanation to why it is correct in
all cases.

\subsection*{Graham Scan}

The algorithm is split into two parts. First it computes the top half of the
convex and then the bottem half. The function \textit{convex_hull_vside}
computes the given vertical side (top or bottem) of the convex hull. For the
function \textit{convex_hull_vside} to work properly, it is necessary that
the points are sorted lexographicly in ascending order. The function
\textit{convex_hull_vside} has a loop that goes through all of the points. We
will show that this loop has the invariant that \textit{convex_hull} is the set
of vertices of the given vertical side of the convex hull of the set of points the
loop has passed. From now on we will call ``the set of vertices of the given vertical side of the convex hull of the set of points the 
loop has passed" the intermediate set. At the end of the loop it has passed through all of the points,
hence it returns the given vertical side of the vertices of the convex hull.

We will now show that the invariant is correct. We know that if there are only two
points or less points, that all the points are vertices of the convex hull, hence
we can add them inmediately to the intermediate set. We know that the new point has to be the point with
the maximum $x$ postion of all points seen so far. 
This implies that the point must be in the intermediate set. We know that a point is not part of the top vertices
of the convex hull if and only if it is below a line or on a line segment going through two other points. Because our invariant
tells us that the previous intermediate set is correct, we only have to check line segments with the new point as an endpoint.
Let $r$ be the new point and $p,q$ be previous two points in the intermediate set. If the line $pr$ goes through or over $q$
we now tat $q$ cannot be in the intermediate convex hull, hence it is removed.
Now look at the case that $pr$ goes under $q$. We now that all remaining points
in the intermediate
set have a x position smaller or equal than that of $p$ and $q$. We now that $q$ lies below above segment $pr$ so the set
of points that have an $x$ position smaller or equal to $p$ and are below the line $pr$ is a subset of the points
with a $x$ position smaller or equal to $p$ that lie below the line $pq$. Because of the invariant we know that all other
points in the intermediate set lies below $pr$, so they must also lie below $pq$, hence all other points remain in the
intermediate set.

\todo{Frank Staals said: ``You will likely run in to an additional issue that, in the theory part of the course, we mostly ignore. See for example what happens when all input points are almost on a line. Think about a way to avoid this issue, and state in your paper what you did to tackle it." I have not found this issue yet.}



\subsection*{Jarvis March}
\subsection*{Divide and Conquer}
The algorithm is a classic divide and conquer algorithm in the sense that the (sorted) imput is split into 2 parts, both of those are once again split recursively until only single points are left.
The algorithm then computes the convex hull of the single point sets, and recursively merges the sets again maintaining a correct convex hull of the subset after every merge.
The merging is done by first finding the upper and lower combined tangent of the two convex hulls to be merged.
When considering the combined shape of the two convex hulls with the two tangents, any point on the interior of this shape is removed to create the combined convex hull.
\\
Most of the logic is in finding the upper and lower tangents for the merging process.
Let $A$ and $B$ be two convex polygons representing two subset convex hulls to be merged. Every point in $B$ is right of every point in $A$ due to the dividing of the points.
The tangent is then found by starting at a vertex pair $a \in A$ and $b \in B$, where the initial $a$ is the rightmost point of $A$ and the initial $b$ the leftmost of $B$. 
Then the candidate tangent $\overline{ab}$ is created.
If both the neighbours of both $a$ and $b$ are to the right of $\overline{ab}$, this candidate is the tangent we were looking for. 
If one of the neighbours of $a$ is on the wrong side, $a$ is updated to be the next vertex of $A$, repeating this process while walking around the polygon. 
The same is done for $b$.
Colinearity is handled by picking the longest possible line segment as the tangent.
\subsection*{Chan's Algorithm}

\section*{Test cases}

\section*{Results}

\section*{Conclusion}

\end{document}
