\documentclass{article}
\usepackage[margin=1in]{geometry}
\usepackage{xcolor}
\usepackage{hyperref}
\usepackage{parskip}

\hypersetup{
	colorlinks=true,
	urlcolor=blue
}
\newcommand{\todo}[1]{\textbf{\textcolor{red}{Todo: #1}}}


\title{Reporting on \href{https://github.com/tvmaaren/GeoAlgBonus-Yanidsmas}{Yanidsmas}.}
\author{Yanna \todo{surname} (\todo{student number}), Ids de Vlas (\todo{student number}), Thomas van Maaren (9825827)}


\begin{document}

\maketitle

In this report we will discuss the programme Yanidsmas. This is a programme that can
given any set of points in the plane, calculate the vertices of it's convex hull. The
user has to choose between four algorithms: Graham Scan, Jarvis March, Divide
and Conquer and Chan's Algorithm. The programme can be cloned be from our
\href{https://github.com/tvmaaren/GeoAlgBonus-Yanidsmas}{git repository}.

\section*{Language and Environment}

We have decided to use Rust to write Yanidsmas. This is because we wanted an imperative
language that gives a very low chance of runtime errors or unpredictable behaviour.
Our git repository contains a \href{https://github.com/tvmaaren/GeoAlgBonus-Yanidsmas/tree/main/tests}{tests} directory that contains all the test cases we came up with.
When you run the program on a testcase it will output a png image. This image shows
a scatter plot of the points in the test case where every vertex of the Convex hull is
marked red. We did this to make it possible to see at a glance if the programme gives
the right output. When benchmarking the scatter plot is disabled as the
time this takes is not relevant to measure the speed of our algorithms. For a more
detailed explanation on how to use Yanidsmas we refer to README.md file in our git repository.

\section*{Correctness of the algorithms}



\section*{Test cases}

\section*{Results}

\section*{Conclusion}

\end{document}
